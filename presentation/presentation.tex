% rubber: module pdftex
% rubber: path HY

%%%%%%%%%%%%%%%%%%%%%%%%%%%%%%%%%%%%%%%%%%%%%%%%%%%%%%%%%%
%  Usage HOWTO                                           %
%  -----------                                           %
%  * Compilation:          rubber presentation           %
%  * Cleaning:             rubber --clean presentation   %
%  * Force recompilation:  rubber --force presentation   %
%%%%%%%%%%%%%%%%%%%%%%%%%%%%%%%%%%%%%%%%%%%%%%%%%%%%%%%%%%

\documentclass[t,12pt,pdftex]{beamer}
\usepackage{helvet}
\usepackage{times}
\usepackage{courier}

\usepackage[T1]{fontenc}
\usepackage[english]{babel}

\usepackage{amssymb}
\usepackage{amsmath}
\usepackage{amsfonts}
\usepackage{graphicx}
\usepackage{color}
\usepackage{url}
\usepackage{textpos}
\usepackage{xspace}
\usepackage{array}

\graphicspath{{./fig/}}

% theme options: hy/ml/hum, rovio/sinetti, hiit
% default: hy,rovio

%\usetheme[hy]{HY}
%\usetheme[hy,sinetti]{HY}
%\usetheme[hum,rovio]{HY}
\usetheme[ml,rovio]{HY}
%\usetheme[ml,rovio,hiit]{HY}


\title{The Title of Presentation}
%\author{Name Name \scriptsize \guilsinglleft{}name.name@helsinki.fi\guilsinglright{}}
\author{Author1, Author2}
\institute{University of Helsinki\\Department of Computer Science}
\date{December 21, 2012}

\begin{document}

%\selectlanguage{english}

\HyTitle

\begin{frame}
	\frametitle{Outline}
	\tableofcontents
\end{frame}


%\AtBeginSection[]
%{
%  \begin{frame}<beamer>
%    \frametitle{Outline}
%    \tableofcontents[currentsection]
%  \end{frame}
%}

\section{Introduction}

\begin{frame}
	\frametitle{One slide}
	
	\begin{itemize}
		\item one point
		\item another point
	\end{itemize}
\end{frame}

\begin{frame}
	\frametitle{Another slide}
	
	asdf
\end{frame}

\section{Conclusion}

\begin{frame}
	\frametitle{Yet another slide}
	
	content
\end{frame}

\end{document}
